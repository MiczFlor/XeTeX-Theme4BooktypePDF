% this document sets all the variables used for metadata in this book

% GENERAL SETTINGS
\newcommand{\varBookLanguage}       {english}
% english, german
\newcommand{\varBookTheme}          {bodde1}
% available themes:
% default, bauhaus, victoriannovel, gothicflower

% FRONT MATTER
\newcommand{\varShowHalfTitle}      {true} % (true|false)
\newcommand{\varShowTitlePage}      {true} % (true|false)
\newcommand{\varShowColophon}       {true} % (true|false)
\newcommand{\varShowDedication}     {true} % (true|false)
\newcommand{\varShowToContents}     {true} % (true|false)
\newcommand{\varShowToFigures}      {true} % (true|false) opt. overwritten by variables-theme.tex
\newcommand{\varShowToTables}       {false} % (true|false) opt. overwritten by variables-theme.tex

% PAPER FORMAT AND MARGINS
% use cm, mm or in (for inch). for more info:
% https://en.wikibooks.org/wiki/LaTeX/Lengths
\newcommand{\varPaperWidth}         {5in} 
\newcommand{\varPaperHeight}        {8in} 
\newcommand{\varBindingOffset}      {0.76cm} 
\newcommand{\varMarginTop}          {1.93cm} % measurement excluding header!
\newcommand{\varMarginBottom}       {1.93cm}
\newcommand{\varMarginLeft}         {1.52cm}
\newcommand{\varMarginRight}        {1.52cm}

% METADATA
% the collophon will print all filled in information
% if you want to hide something, leave it empty like this: {}
% IMPORTANT: all variables here must be initiated, do not comment them out
\newcommand{\varTitle}            {Dracula meets Frankenstein}
\newcommand{\varSubtitle}         {What happens when two worlds collide?}
\newcommand{\varAuthors}          {Author van Book}
\newcommand{\varURLAuthor}        {www.authorwebsite.com}
\newcommand{\varDedication}       {Dedicating this to Somebody Special

And another line of dedication, because there is always somebody else you should have mentioned---and they will never, ever forgive you.}
\newcommand{\varShorttitle}       {}
\newcommand{\varCopyrightdate}    {2007, 2008, 2012}
\newcommand{\varCopyrightholder}  {Copy R. Holder}

\newcommand{\varPublicationdate}   {1st October 2010}
\newcommand{\varPublisher}         {Publishing Company Ltd.}
\newcommand{\varPublishercity}     {Berlin}
\newcommand{\varURLPublisher}      {www.publishersite.com}

\newcommand{\varCoverdesign}			{C. O. ver Designer}
\newcommand{\varCoverimage}				{Image Cover}
\newcommand{\varPhotographyby}		{Photo G. Rapher}

\newcommand{\varShortdescription} {}
\newcommand{\varLongdescription}  {}
\newcommand{\varEbookISBN}        {0 123 12345 1}
\newcommand{\varPrintISBN}        {9 876 54321 0}
\newcommand{\varEditedby}         {Edward Itor}
\newcommand{\varTextby}           {Tex T. Writer}
\newcommand{\varTranslationby}    {Babel Fish}
\newcommand{\varIntroductionby}   {Intro Ducer}
\newcommand{\varIllustrationby}   {Illu Stration}
\newcommand{\varResearch}         {R. E. Search}
\newcommand{\varLectorate}        {Hanibal Lector}
\newcommand{\varProofreading}     {Proof U. Need}
\newcommand{\varRightsclearing}   {Rechte Klar}
\newcommand{\varTypeface}         {Ubuntu Type Font Version 4.2 2015}

\newcommand{\varPrintercompany}     {International Printers}
\newcommand{\varPrintercountry}     {Denmark}
\newcommand{\varPrintedon}          {White Recycling Paper}
\newcommand{\varPapercertification} {This paper is climate neutral made from recycled materials and not bleached.}
\newcommand{\varBookbinder}         {Book The Binder}
\newcommand{\varAcknowledge}        {Acknowledgement In the creative arts and scientific literature, an acknowledgment (also spelled acknowledgement) is an expression of gratitude for assistance in creating an original work.In the creative arts and scientific literature, an acknowledgment (also spelled acknowledgement) is an expression of gratitude for assistance in creating an original work.

In the creative arts and scientific literature, an acknowledgment (also spelled acknowledgement) is an expression of gratitude for assistance in creating an original work.}
\newcommand{\varEdition}                  {3rd edition}
\newcommand{\varBibliographicinformation} {Bibliographic information.In the creative arts and scientific literature, an acknowledgment (also spelled acknowledgement) is an expression of gratitude for assistance in creating an original work.}
\newcommand{\varCreationTool}             {This book was created, edited and prepared for print using Booktype, the browser based book authoring platform. www.booktype.pro}

% FIGURES (images)
% set some variables, not to be used, but to reset with \renewcommand without errors
% variables can be assigned for each image
% NOTE: the below values are not default for images.
% If \recommand in LaTeX the value will change as of then for the rest of the text
\newcommand{\varWrapText}     {false} % (true|false) wrap text around image?
\newcommand{\varFigWidth}     {scale} % (text|paper|scale|0.x) 
% 0.x like 0.5 = half text width
% scale = max width is text
% text = force width of text
% paper = force width of paper
\newcommand{\varFigFloat}     {l} % (r|R|l|L|i|I|o|O) only if \varWrapText http://ctan.org/pkg/wrapfig
\newcommand{\varCaption}      {} % Caption. \renewcommand each image in text
\newcommand{\varImgPath}      {} % Path To Image. \renewcommand each image in text
% global for document (i.e. don't change inside text)
\newcommand{\varFigPlacement} {hpb} % leave unchanged if you don't know what you do
