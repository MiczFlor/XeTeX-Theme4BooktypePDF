\chapter{Introduction}
"Sourcefabric? Cooking? I thought they were all about about open source tools for journalism." - you may wonder. And you are right. Which is why we find ourselves often in a position to explain the benefits of open source development. From all the metaphors we came up with, cooking and recipies in particular were the best.
Sourcefabric brings together over 60 people from over 20 countries. This is reflected in the collection of recipies we have whipped together here. What better way to celebrate diversity than with a delicious meal.
\section{Open Sauce explaining Open Source}
Let us walk you through three of the ways in which recipies are a recipe for open source development.
\subsection{Sharing resources}
You want to cook a heart and belly warming stew and find yourself with only two potatoes in the basket. What are you going to do? In an open source environment, you would reach out to the community and explain your idea, tell everybody what you bring to the table. Others than can decide to join the effort, bring their pots, pans and vegetables and pool resources. Chances are that you change your ideas about the kind of stew you were planning to make. This is because you will encounter vegetables which you had never seen before and ways of preparing the stew that seem to work better than what you originally had come up with. And in the end, well, there is enough for everybody. And there is more.
\subsection{Transparency}
Did you ever wonder how bread came into this world? What an elaborate process. Harvesting seeds, drying them, grinding them, mixing them with water and salt. In some cultures you are using sour dough to make the doug rise, some use soda, some yeast, some nothing at all and just create flat sheets to wrap around other food. Sharing trials and errors, methods and ideas over thousands of years allowed the creation of bread. And not just one bread, but many regional versions with unique tastes and actually adjusted to regional ingredients as well as the environment (in some regions, for example, bread has to be very dry to last). 
\subsection{Diversity}
 
\chapter{Pälzer Schobbe}
by Fabienne Riener (the Pfalzprinzessin)
Coming from the most beautiful wine region in Germany - the Pfalz - my suggestion if of course: The Pälzer Woi Schobbe. 
If you can't be bothered reading the below, if you are open to some cultural stereotyping and/or if you are curious about the Palation accent, watch this highly entertaining video about how to make a real Pälzer Woi Schobbe.
This is what the finished product looks like:

\subsubsection{Headline 4}
\paragraph{Headline 5}
\subparagraph{Headline 6}
It's what fuels the region and its people and it's so easy to make that even 10 year olds know how to make it for their mums and dads. 
The Pfalz is the second largest wine area in Germany with 3600 winemakers looking after over 100 Million vines. It's serious business and a trip to a "Weinstube" (Wine Pub) is a must for every visitor. 
On with the Schobbe. All you need are 3 ingredients:

The Schobbeglas 
A guhdhe Pälzer Woi
Sprudel


\section{1) The Schobbe Glas / Schoppe / Dubbeglas / Pälzer Dorschdmesser}
It's basically a wine glas. But it's very, very large. It holds 500ml (that's half a liter / slightly less than a whole pint) and always has to be as full as possible,
See here and note the dents on the side for an easy grip.


Every Pfalz household has to have at least 6 of those at home. If you don't, you are not a real Pfälzer.

\section{2) A guhe Pälzer Woi = Good Palatian Wine}
That's usally Riesling.


Grauburgunder (Pinot Grigio) or Weißburgunder (Pino White) are also good, but these are better drank as "normal wine", in a 250ml glas, without any Sprudel. 
Here you can buy good Palatian Wine (I am not affiliated to this business!). 

\section{3) Sprudel = sparkling water}


Any brand is ok. 

Let's bring them together:
1)  Take the Schobbeglas
2) Open the Wine bottle
3) Pour the glas full to the top - I mean it - to the top. 
4) Take a small sip of wine. 
5) Open the Water bottle
6) Refill the Schobbeglas with water, again until it's full to the top.
DONE!

Prost! Or, as we say in the Pfalz:  





\chapter{Handkäs mit Musik}
by Micz Flor
"Handkäs mit Musik" (yes, you guessed right: hand cheese with music) comes from the region in Germany where I grew up: Hessen. (You might have heard of Hessen in relation to "The Hessian" of Sleepy Hollow.)


Wikipedia does give a wonderful introduction to the recipe: 
Handkäse  is a German regional sour milk cheese (similar to Harzer) and is a culinary speciality of Frankfurt am Main, Offenbach am Main, Darmstadt, Langen and all other parts of southern Hesse. It gets its name from the traditional way of producing it: forming it with one's own hands.
It is a small, translucent, yellow cheese with a pungent aroma that many people find unpleasant. It is sometimes square, but more often round in shape.
Often served as an appetizer or as a snack with Apfelwein (aka Ebbelwoi or cider), it is traditionally topped with chopped onions,locally known as "Handkäse mit Musik" (literally: hand cheese with music). It is usually eaten with caraway on it, but since many people in Germany do not like this spice, in many areas it is served on the side. Some Hessians say that it is a sign of the quality of the establishment when caraway is in a separate dispenser. As a sign of this many restaurants have, in addition to the salt and pepper, a little pot for caraway seeds.
Strangers to this custom will probably ask where the "Musik" is. They will most likely be told that "Die Musik kommt später," i.e. the music "comes later." This is a euphemism for the flatulence that the raw onions usually provide. A more polite, but less likely explanation for the "Musik" is that the flasks of vinegar and oil customarily provided with the cheese would strike a musical note when they hit each other. Handkäse is popular among dieters and some health food devotees. It is also popular among bodybuilders, runners, and weightlifters for its high content of protein while being relatively low in fat.
And this is how you do it:
4 Handkäse Cheeses
16 tbsp Vinegar (white)
4 tbsp Cider (Apfelwein, acutally)
4 tbsp Oil
4 Diced Onions (diced)

Mix all the ingredients together and use as a marinade. Pour it on the cheese and let it sit for some time (the longer the juicier). Serve with Sauerteigbrot (Roggenmisch sour sough bread) and butter - and caraway on the side. (Some say "add cumin" but that's not right, I think. Cumin in German is Kreuzkümmel and that's a different kind. I guess caraway is the correct term).
