\chapter{Chapter title and text examples}
Paragraph style "Normal Text"\lipsum[2]

\begin{quote}
Paragraph style "Quote" Lorem ipsum dolor sit
amet, consetetur sadipscing elitr, sed diam nonumy (press
Shift+Return for single line breaks)\\
eirmod tempor invidunt
\end{quote}

Now we will do some font size. Booktype gives me the
option to select one of the following font sizes:

\font\FontSizeTenPT="Ubuntu" at 10pt
\font\FontSizeTwelvePT="Ubuntu" at 12pt
\font\FontSizeFourteenPT="Ubuntu" at 14pt
\font\FontSizeSixteenPT="Ubuntu" at 16pt
\font\FontSizeEighteenPT="Ubuntu" at 18pt
\font\FontSizeTwentyPT="Ubuntu" at 20pt
\font\FontSizeTwentytwoPT="Ubuntu" at 22pt
\font\FontSizeTwentyfourPT="Ubuntu" at 24pt
\font\FontSizeTwentysixPT="Ubuntu" at 26pt
\font\FontSizeTwentyeightPT="Ubuntu" at 28pt
\font\FontSizeThirtyPT="Ubuntu" at 30pt

{\FontSizeTenPT 10pt is the size of this paragraph.}

{\FontSizeTwelvePT 12pt is the size of this paragraph.}

{\FontSizeFourteenPT 14pt is the size of this paragraph.}

{\FontSizeSixteenPT 16pt is the size of this paragraph.}

{\FontSizeEighteenPT 18pt is the size of this paragraph.}

{\FontSizeTwentyPT 20pt is the size of this paragraph.}

{\FontSizeTwentytwoPT 22pt is the size of this paragraph.}

{\FontSizeTwentyfourPT 24pt is the size of this paragraph.}

{\FontSizeTwentysixPT 26pt is the size of this paragraph.}

{\FontSizeTwentyeightPT 28pt is the size of this paragraph.}

{\FontSizeThirtyPT 30pt is the size of this paragraph.}

\chapter{Style "Heading" and colours}

Paragraph style "Normal Text" with a couple of font
styles. Here is 
\textbf{some bold text}. And here is 
\textit{some italicised text}. 
Part of this sentence is underlined. 
We can also combine these, starting 
\textit{italicise now, \textbf{then some bold}, back to italics only} and back to normal. This can
also happen the other way around: 
\textbf{starting bold now, \textit{then some italics}, back to bold only} and
back to normal.

Here is a coloured piece of text: 
\textcolor[rgb]{1,0,0}{This is \textbf{red}.}
\textcolor[rgb]{0,0,1}{And this is \textbf{blue}.}
\textcolor[rgb]{0,1,0}{This is \textit{green}.}
\textcolor[rgb]{0.5,0.5,0.5}{\textbf{And pale are you}.}

\lipsum[3]

\section{Style "Subheading", tables and links}
\lipsum[2]
And let's now insert a table:

% http://www.andy-roberts.net/writing/latex/tables
% max width == textwidth, see: http://tex.stackexchange.com/questions/121155/how-to-adjust-a-table-to-fit-on-page
\noindent
\begin{table}[H]
	\begin{center}
   \small
   \begin{adjustbox}{max width=\textwidth}
	% hide table numbering if set for list of tables or chapters in general
	% set in theme variables
	\ifnum\pdfstrcmp{\varShowToTablesNumbering}{false}=0 
		\caption*{Just a few names in this table}
	\else
		\ifnum\pdfstrcmp{\varChapterWriteNumbering}{false}=0 
			\caption*{Just a few names in this table}
		\else
			\caption{Just a few names in this table} 
		\fi
	\fi
		\begin{tabular}{lllll}
			\toprule
			Feature & Icon & Description & HTML & LaTeX \\
			\midrule
			Bold & Fat B & Makes the font look fatter & <b> or <strong> & textbf \\
			Italicise & Italics I & Font leans to the right & <i> or <emph> & textit \\
			Underline & U with a line beneath & Draws a line under the text & text-decoration: underline & underline \\
			\bottomrule
		\end{tabular}
   \end{adjustbox}
	\end{center}
\end{table}

Some more text, then another, smaller table. \lipsum[3]

% http://www.andy-roberts.net/writing/latex/tables
% max width == textwidth, see: http://tex.stackexchange.com/questions/121155/how-to-adjust-a-table-to-fit-on-page
\noindent
\begin{table}[H]
	\begin{center}
   \small
   \begin{adjustbox}{max width=\textwidth}
	% hide table numbering if set for list of tables or chapters in general
	% set in theme variables
	\ifnum\pdfstrcmp{\varShowToTablesNumbering}{false}=0 
		\caption*{Even fewer names in this table}
	\else
		\ifnum\pdfstrcmp{\varChapterWriteNumbering}{false}=0 
			\caption*{Even fewer names in this table}
		\else
			\caption{Even fewer names in this table} 
		\fi
	\fi
		\begin{tabular}{lllll}
			\toprule
			Feature & Icon & Description \\
			\midrule
			Bold & Fat B & Makes the font look fatter \\
			Italicise & Italics I & Font leans to the right \\
			Underline & U with a line beneath & Draws a line under the text \\
			\bottomrule
		\end{tabular}
   \end{adjustbox}
	\end{center}
\end{table}

\subsubsection{Links}
Let's not forget about links. The links work, but they are displayed in the same
style as other text.
Here's a link to \href{https://booktype.pro/}{booktype.pro}. 
Booktype allows authors to create beautiful books for print and digital
distribution. Find out 
\href{https://booktype.pro/en/booktype/screenshots/}{more on our
features page}.

\lipsum[3]

\subsection{Style "SubSubheading", lists, align and endnotes}
\footnote{This is an endnote at the beginning
of a paragraph, something that would never actually happen. But
might.}Sed diam
nonumy eirmod tempor invidunt ut labore et dolore magna aliquyam
erat, sed diam voluptua. At vero eos et accusam et justo duo
dolores et ea rebum\footnote{This is an endnote in the middle of
a paragraph.}. Stet clita kasd gubergren, no sea takimata
sanctus est Lorem ipsum dolor sit amet.\footnote{This is an endnote at the end of the
paragraph.}

A list of ordered things:

\begin{enumerate}  
\item the first item on the list 
\item the second item on the list 
\item the third item on the list\\a forced line break within the list
\item and the last item on the list
\end{enumerate}

And here is an unorderly list

\begin{itemize}
\item A bullet point item could be anywhere in the list
\item The same goes for this item with a\\forced line break inside
\item And the last item is not telling us anything useful either
\end{itemize}

\begin{flushleft}
Left aligned text. \lipsum[1]
\end{flushleft}

\begin{flushright}
Right aligned text. \lipsum[1]
\end{flushright}

\begin{center}
Center aligned text. \lipsum[1]
\end{center}

Justify /
Blocksatz text. \lipsum[1]

\subsection{Indentation, image and preformatted text}

\begin{adjustwidth}{2cm}{}Did I mention
that there can be indents? This paragraph is indented once.
\end{adjustwidth}

\begin{adjustwidth}{4cm}{}This paragraph
is indented twice.
\end{adjustwidth}

\lipsum[3]

\texttt{Style "preformatted" Sed diam nonumy eirmod tempor invidunt ut \\Shift+Enter (single line break) labore et dolore magna\\  Enter (double line break)}
\texttt{Shift+Enter (single line break) aliquyam erat, }
\texttt{   Shift+Enter (single line break) sed diam voluptua. At vero eos et accusam et justo duo dolores et ea rebum.}

I am adding an image here:

\renewcommand{\varCaption}{This is an image with the caption: There are three minutes left on the clock.}
\renewcommand{\varImgPath}{_content/images/3minutes700x400.jpg}
% calc image measurements
% this is for later to calculate print sizes...
%\settoheight\imageheight{\includegraphics{\varImgPath}}
%\settowidth\imagewidth{\includegraphics{\varImgPath}}
%Image Original Height: \the\imageheight\ (\printlength{\imageheight}) Original Width: \the\imagewidth\ (\printlength{\imagewidth})

% caption settings
\captionsetup[figure]{	
	font=\varFigCapLabelSize,
}
% hide numbers if set that way in theme variables
\ifnum\pdfstrcmp{\varShowToFiguresNumbering}{false}=0
\captionsetup[figure]{
		labelformat=empty,
}
\fi
% hide numbers if chapter numbers are hidden set that way in theme variables
\ifnum\pdfstrcmp{\varChapterWriteNumbering}{false}=0
\captionsetup[figure]{
		labelformat=empty,
}
\fi

% to see \linewidth
% \hrule

%\begin{figure}
%\includegraphics[max width=\linewidth]{example-image-1x1}
%\end{figure}
%\begin{figure}
%\includegraphics[max width=\linewidth]{example-image-a4}
%\end{figure}


\ifnum\pdfstrcmp{\varWrapText}{true}=0 
	% wrap text around figure
	\begin{wrapfigure}{\varFigFloat}{\varFigWidth\textwidth}
		\vspace{-12pt} % move image up a bit to align correctly to top
		\includegraphics[width=\varFigWidth\textwidth]{\varImgPath} 
		% if caption empty, don't print it (no empty line)
		\ifnum\pdfstrcmp{\varCaption	}{}=0 \else \caption{\varCaption} \fi 
	\end{wrapfigure}
\else
	% do not wrap text around figure
	\begin{figure}[\varFigPlacement]
	\centering
	\ifnum\pdfstrcmp{\varFigWidth}{paper}=0 
		% image is paperwidth
		% max height a little less than text height to incluce caption
		\includegraphics[width=\paperwidth, max height=0.95\textheight]{\varImgPath}
	\else
		\ifnum\pdfstrcmp{\varFigWidth}{text}=0 
			% image is text width
			% max height a little less than text height to incluce caption
			\includegraphics[width=\textwidth, max height=0.95\textheight]{\varImgPath}			
		\else
			\ifnum\pdfstrcmp{\varFigWidth}{scale}=0 
				% image is as big as possible, max width = text width
				% max height a little less than text height to incluce caption
				\includegraphics[max width=\textwidth, max height=0.95\textheight]{\varImgPath}			
			\else
				% image is scaled according to parameter given in \varFigWidth
				% max height a little less than text height to incluce caption
				\includegraphics[max width=\varFigWidth\textwidth, max height=0.95\textheight]{\varImgPath}
			\fi 
		\fi 
	\fi 
	% if caption empty, don't print it (no empty line)
	\ifnum\pdfstrcmp{\varCaption}{}=0 \else \caption{\varCaption} \fi 
	\end{figure}
\fi 

And then the text continues here.
\lipsum[3]
 
\chapter{Style "Heading" with a lot of text, a lot of text, a lot of
text, and even more text}
Paragraph style "Normal Text" \lipsum[1]
\section{Style "Subheading" with a lot of text, a lot of text, a
lot of text, and even more text}
\lipsum[1]
\subsection{Style "SubSubheading" with a lot of text, a lot of text, a
lot of text, and even more text}
\lipsum[1]