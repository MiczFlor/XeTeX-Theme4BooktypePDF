\chapter{UTF-8 encoded sample and escape characters}

% http://tex.stackexchange.com/questions/34580/escape-character-in-latex#34586
\section{Escape characters}
\& \% \$ \# \_ \{ \}
\section{Unicode Blocks}

\subsection{Latin Extended-A}
\url{https://en.wikipedia.org/wiki/Latin_Extended-A}

U+010x 	Ā 	ā 	Ă 	ă 	Ą 	ą 	Ć 	ć 	Ĉ 	ĉ 	Ċ 	ċ 	Č 	č 	Ď 	ď

U+011x 	Đ 	đ 	Ē 	ē 	Ĕ 	ĕ 	Ė 	ė 	Ę 	ę 	Ě 	ě 	Ĝ 	ĝ 	Ğ 	ğ

U+012x 	Ġ 	ġ 	Ģ 	ģ 	Ĥ 	ĥ 	Ħ 	ħ 	Ĩ 	ĩ 	Ī 	ī 	Ĭ 	ĭ 	Į 	į

U+013x 	İ 	ı 	IJ 	ij 	Ĵ 	ĵ 	Ķ 	ķ 	ĸ 	Ĺ 	ĺ 	Ļ 	ļ 	Ľ 	ľ 	Ŀ

U+014x 	ŀ 	Ł 	ł 	Ń 	ń 	Ņ 	ņ 	Ň 	ň 	ʼn 	Ŋ 	ŋ 	Ō 	ō 	Ŏ 	ŏ

U+015x 	Ő 	ő 	Œ 	œ 	Ŕ 	ŕ 	Ŗ 	ŗ 	Ř 	ř 	Ś 	ś 	Ŝ 	ŝ 	Ş 	ş

U+016x 	Š 	š 	Ţ 	ţ 	Ť 	ť 	Ŧ 	ŧ 	Ũ 	ũ 	Ū 	ū 	Ŭ 	ŭ 	Ů 	ů

U+017x 	Ű 	ű 	Ų 	ų 	Ŵ 	ŵ 	Ŷ 	ŷ 	Ÿ 	Ź 	ź 	Ż 	ż 	Ž 	ž 	ſ

\subsection{Latin Extended-B}
\url{https://en.wikipedia.org/wiki/Latin_Extended-B}

U+018x 	ƀ 	Ɓ 	Ƃ 	ƃ 	Ƅ 	ƅ 	Ɔ 	Ƈ 	ƈ 	Ɖ 	Ɗ 	Ƌ 	ƌ 	ƍ 	Ǝ 	Ə

U+019x 	Ɛ 	Ƒ 	ƒ 	Ɠ 	Ɣ 	ƕ 	Ɩ 	Ɨ 	Ƙ 	ƙ 	ƚ 	ƛ 	Ɯ 	Ɲ 	ƞ 	Ɵ

U+01Ax 	Ơ 	ơ 	Ƣ 	ƣ 	Ƥ 	ƥ 	Ʀ 	Ƨ 	ƨ 	Ʃ 	ƪ 	ƫ 	Ƭ 	ƭ 	Ʈ 	Ư

U+01Bx 	ư 	Ʊ 	Ʋ 	Ƴ 	ƴ 	Ƶ 	ƶ 	Ʒ 	Ƹ 	ƹ 	ƺ 	ƻ 	Ƽ 	ƽ 	ƾ 	ƿ

U+01Cx 	ǀ 	ǁ 	ǂ 	ǃ 	DŽ 	Dž 	dž 	LJ 	Lj 	lj 	NJ 	Nj 	nj 	Ǎ 	ǎ 	Ǐ

U+01Dx 	ǐ 	Ǒ 	ǒ 	Ǔ 	ǔ 	Ǖ 	ǖ 	Ǘ 	ǘ 	Ǚ 	ǚ 	Ǜ 	ǜ 	ǝ 	Ǟ 	ǟ

U+01Ex 	Ǡ 	ǡ 	Ǣ 	ǣ 	Ǥ 	ǥ 	Ǧ 	ǧ 	Ǩ 	ǩ 	Ǫ 	ǫ 	Ǭ 	ǭ 	Ǯ 	ǯ

U+01Fx 	ǰ 	DZ 	Dz 	dz 	Ǵ 	ǵ 	Ƕ 	Ƿ 	Ǹ 	ǹ 	Ǻ 	ǻ 	Ǽ 	ǽ 	Ǿ 	ǿ

U+020x 	Ȁ 	ȁ 	Ȃ 	ȃ 	Ȅ 	ȅ 	Ȇ 	ȇ 	Ȉ 	ȉ 	Ȋ 	ȋ 	Ȍ 	ȍ 	Ȏ 	ȏ

U+021x 	Ȑ 	ȑ 	Ȓ 	ȓ 	Ȕ 	ȕ 	Ȗ 	ȗ 	Ș 	ș 	Ț 	ț 	Ȝ 	ȝ 	Ȟ 	ȟ

U+022x 	Ƞ 	ȡ 	Ȣ 	ȣ 	Ȥ 	ȥ 	Ȧ 	ȧ 	Ȩ 	ȩ 	Ȫ 	ȫ 	Ȭ 	ȭ 	Ȯ 	ȯ

U+023x 	Ȱ 	ȱ 	Ȳ 	ȳ 	ȴ 	ȵ 	ȶ 	ȷ 	ȸ 	ȹ 	Ⱥ 	Ȼ 	ȼ 	Ƚ 	Ⱦ 	ȿ

U+024x 	ɀ 	Ɂ 	ɂ 	Ƀ 	Ʉ 	Ʌ 	Ɇ 	ɇ 	Ɉ 	ɉ 	Ɋ 	ɋ 	Ɍ 	ɍ 	Ɏ 	ɏ

\subsection{Latin Extended Additional}
\url{https://en.wikipedia.org/wiki/Latin_Extended_Additional}

U+1E0x 	Ḁ 	ḁ 	Ḃ 	ḃ 	Ḅ 	ḅ 	Ḇ 	ḇ 	Ḉ 	ḉ 	Ḋ 	ḋ 	Ḍ 	ḍ 	Ḏ 	ḏ

U+1E1x 	Ḑ 	ḑ 	Ḓ 	ḓ 	Ḕ 	ḕ 	Ḗ 	ḗ 	Ḙ 	ḙ 	Ḛ 	ḛ 	Ḝ 	ḝ 	Ḟ 	ḟ

U+1E2x 	Ḡ 	ḡ 	Ḣ 	ḣ 	Ḥ 	Ḥ 	Ḧ 	ḧ 	Ḩ 	ḩ 	Ḫ 	ḫ 	Ḭ 	ḭ 	Ḯ 	ḯ

U+1E3x 	Ḱ 	ḱ 	Ḳ 	ḳ 	Ḵ 	ḵ 	Ḷ 	ḷ 	Ḹ 	ḹ 	Ḻ 	ḻ 	Ḽ 	ḽ 	Ḿ 	ḿ

U+1E4x 	Ṁ 	ṁ 	Ṃ 	ṃ 	Ṅ 	ṅ 	Ṇ 	ṇ 	Ṉ 	ṉ 	Ṋ 	ṋ 	Ṍ 	ṍ 	Ṏ 	ṏ

U+1E5x 	Ṑ 	ṑ 	Ṓ 	ṓ 	Ṕ 	ṕ 	Ṗ 	ṗ 	Ṙ 	ṙ 	Ṛ 	ṛ 	Ṝ 	ṝ 	Ṟ 	ṟ

U+1E6x 	Ṡ 	ṡ 	Ṣ 	ṣ 	Ṥ 	ṥ 	Ṧ 	ṧ 	Ṩ 	ṩ 	Ṫ 	ṫ 	Ṭ 	ṭ 	Ṯ 	ṯ

U+1E7x 	Ṱ 	ṱ 	Ṳ 	ṳ 	Ṵ 	ṵ 	Ṷ 	ṷ 	Ṹ 	ṹ 	Ṻ 	ṻ 	Ṽ 	ṽ 	Ṿ 	ṿ

U+1E8x 	Ẁ 	ẁ 	Ẃ 	ẃ 	Ẅ 	ẅ 	Ẇ 	ẇ 	Ẉ 	ẉ 	Ẋ 	ẋ 	Ẍ 	ẍ 	Ẏ 	ẏ

U+1E9x 	Ẑ 	ẑ 	Ẓ 	ẓ 	Ẕ 	ẕ 	ẖ 	ẗ 	ẘ 	ẙ 	ẚ 	ẛ 	ẜ 	ẝ 	ẞ 	ẟ

U+1EAx 	Ạ 	ạ 	Ả 	ả 	Ấ 	ấ 	Ầ 	ầ 	Ẩ 	ẩ 	Ẫ 	ẫ 	Ậ 	ậ 	Ắ 	ắ

U+1EBx 	Ằ 	ằ 	Ẳ 	ẳ 	Ẵ 	ẵ 	Ặ 	ặ 	Ẹ 	ẹ 	Ẻ 	ẻ 	Ẽ 	ẽ 	Ế 	ế

U+1ECx 	Ề 	ề 	Ể 	ể 	Ễ 	ễ 	Ệ 	ệ 	Ỉ 	ỉ 	Ị 	ị 	Ọ 	ọ 	Ỏ 	ỏ

U+1EDx 	Ố 	ố 	Ồ 	ồ 	Ổ 	ổ 	Ỗ 	ỗ 	Ộ 	ộ 	Ớ 	ớ 	Ờ 	ờ 	Ở 	ở

U+1EEx 	Ỡ 	ỡ 	Ợ 	ợ 	Ụ 	ụ 	Ủ 	ủ 	Ứ 	ứ 	Ừ 	ừ 	Ử 	ử 	Ữ 	ữ

U+1EFx 	Ự 	ự 	Ỳ 	ỳ 	Ỵ 	ỵ 	Ỷ 	ỷ 	Ỹ 	ỹ 	Ỻ 	ỻ 	Ỽ 	ỽ 	Ỿ 	ỿ

\subsection{Greek and Coptic}
\url{https://en.wikipedia.org/wiki/Greek_and_Coptic}
Greek and Coptic is the Unicode block for representing modern (monotonic) Greek. It was originally used for writing Coptic, using the similar Greek letters, in addition to the uniquely Coptic additions. Beginning with version 4.1 of the Unicode Standard, a separate Coptic block has been included in Unicode, allowing for mixed Greek/Coptic text that is stylistically contrastive, as is convention in scholarly works. Writing polytonic Greek requires the use of combining characters or the precomposed vowel + tone characters in the Greek Extended character block.

U+037x 	Ͱ 	ͱ 	Ͳ 	ͳ 	ʹ 	͵ 	Ͷ 	ͷ 			ͺ 	ͻ 	ͼ 	ͽ 	; 	Ϳ

U+038x 					΄ 	΅ 	Ά 	· 	Έ 	Ή 	Ί 		Ό 		Ύ 	Ώ

U+039x 	ΐ 	Α 	Β 	Γ 	Δ 	Ε 	Ζ 	Η 	Θ 	Ι 	Κ 	Λ 	Μ 	Ν 	Ξ 	Ο

U+03Ax 	Π 	Ρ 		Σ 	Τ 	Υ 	Φ 	Χ 	Ψ 	Ω 	Ϊ 	Ϋ 	ά 	έ 	ή 	ί

U+03Bx 	ΰ 	α 	β 	γ 	δ 	ε 	ζ 	η 	θ 	ι 	κ 	λ 	μ 	ν 	ξ 	ο

U+03Cx 	π 	ρ 	ς 	σ 	τ 	υ 	φ 	χ 	ψ 	ω 	ϊ 	ϋ 	ό 	ύ 	ώ 	Ϗ

U+03Dx 	ϐ 	ϑ 	ϒ 	ϓ 	ϔ 	ϕ 	ϖ 	ϗ 	Ϙ 	ϙ 	Ϛ 	ϛ 	Ϝ 	ϝ 	Ϟ 	ϟ

U+03Ex 	Ϡ 	ϡ 	Ϣ 	ϣ 	Ϥ 	ϥ 	Ϧ 	ϧ 	Ϩ 	ϩ 	Ϫ 	ϫ 	Ϭ 	ϭ 	Ϯ 	ϯ

U+03Fx 	ϰ 	ϱ 	ϲ 	ϳ 	ϴ 	ϵ 	϶ 	Ϸ 	ϸ 	Ϲ 	Ϻ 	ϻ 	ϼ 	Ͻ 	Ͼ 	Ͽ


\subsection{Greek Extended}
\url{https://en.wikipedia.org/wiki/Greek_Extended}

U+1F0x 	ἀ 	ἁ 	ἂ 	ἃ 	ἄ 	ἅ 	ἆ 	ἇ 	Ἀ 	Ἁ 	Ἂ 	Ἃ 	Ἄ 	Ἅ 	Ἆ 	Ἇ

U+1F1x 	ἐ 	ἑ 	ἒ 	ἓ 	ἔ 	ἕ 			Ἐ 	Ἑ 	Ἒ 	Ἓ 	Ἔ 	Ἕ 		

U+1F2x 	ἠ 	ἡ 	ἢ 	ἣ 	ἤ 	ἥ 	ἦ 	ἧ 	Ἠ 	Ἡ 	Ἢ 	Ἣ 	Ἤ 	Ἥ 	Ἦ 	Ἧ

U+1F3x 	ἰ 	ἱ 	ἲ 	ἳ 	ἴ 	ἵ 	ἶ 	ἷ 	Ἰ 	Ἱ 	Ἲ 	Ἳ 	Ἴ 	Ἵ 	Ἶ 	Ἷ

U+1F4x 	ὀ 	ὁ 	ὂ 	ὃ 	ὄ 	ὅ 			Ὀ 	Ὁ 	Ὂ 	Ὃ 	Ὄ 	Ὅ 		

U+1F5x 	ὐ 	ὑ 	ὒ 	ὓ 	ὔ 	ὕ 	ὖ 	ὗ 		Ὑ 		Ὓ 		Ὕ 		Ὗ

U+1F6x 	ὠ 	ὡ 	ὢ 	ὣ 	ὤ 	ὥ 	ὦ 	ὧ 	Ὠ 	Ὡ 	Ὢ 	Ὣ 	Ὤ 	Ὥ 	Ὦ 	Ὧ

U+1F7x 	ὰ 	ά 	ὲ 	έ 	ὴ 	ή 	ὶ 	ί 	ὸ 	ό 	ὺ 	ύ 	ὼ 	ώ 		

U+1F8x 	ᾀ 	ᾁ 	ᾂ 	ᾃ 	ᾄ 	ᾅ 	ᾆ 	ᾇ 	ᾈ 	ᾉ 	ᾊ 	ᾋ 	ᾌ 	ᾍ 	ᾎ 	ᾏ

U+1F9x 	ᾐ 	ᾑ 	ᾒ 	ᾓ 	ᾔ 	ᾕ 	ᾖ 	ᾗ 	ᾘ 	ᾙ 	ᾚ 	ᾛ 	ᾜ 	ᾝ 	ᾞ 	ᾟ

U+1FAx 	ᾠ 	ᾡ 	ᾢ 	ᾣ 	ᾤ 	ᾥ 	ᾦ 	ᾧ 	ᾨ 	ᾩ 	ᾪ 	ᾫ 	ᾬ 	ᾭ 	ᾮ 	ᾯ

U+1FBx 	ᾰ 	ᾱ 	ᾲ 	ᾳ 	ᾴ 		ᾶ 	ᾷ 	Ᾰ 	Ᾱ 	Ὰ 	Ά 	ᾼ 	᾽ 	ι 	᾿

U+1FCx 	῀ 	῁ 	ῂ 	ῃ 	ῄ 		ῆ 	ῇ 	Ὲ 	Έ 	Ὴ 	Ή 	ῌ 	῍ 	῎ 	῏

U+1FDx 	ῐ 	ῑ 	ῒ 	ΐ 			ῖ 	ῗ 	Ῐ 	Ῑ 	Ὶ 	Ί 		῝ 	῞ 	῟

U+1FEx 	ῠ 	ῡ 	ῢ 	ΰ 	ῤ 	ῥ 	ῦ 	ῧ 	Ῠ 	Ῡ 	Ὺ 	Ύ 	Ῥ 	῭ 	΅ 	`

U+1FFx 			ῲ 	ῳ 	ῴ 		ῶ 	ῷ 	Ὸ 	Ό 	Ὼ 	Ώ 	ῼ 	´ 	῾ 	
 
 
\subsection{Cyrillic}
\url{https://en.wikipedia.org/wiki/Cyrillic_script_in_Unicode}

U+040x 	Ѐ 	Ё 	Ђ 	Ѓ 	Є 	Ѕ 	І 	Ї 	Ј 	Љ 	Њ 	Ћ 	Ќ 	Ѝ 	Ў 	Џ

U+041x 	А 	Б 	В 	Г 	Д 	Е 	Ж 	З 	И 	Й 	К 	Л 	М 	Н 	О 	П

U+042x 	Р 	С 	Т 	У 	Ф 	Х 	Ц 	Ч 	Ш 	Щ 	Ъ 	Ы 	Ь 	Э 	Ю 	Я

U+043x 	а 	б 	в 	г 	д 	е 	ж 	з 	и 	й 	к 	л 	м 	н 	о 	п

U+044x 	р 	с 	т 	у 	ф 	х 	ц 	ч 	ш 	щ 	ъ 	ы 	ь 	э 	ю 	я

U+045x 	ѐ 	ё 	ђ 	ѓ 	є 	ѕ 	і 	ї 	ј 	љ 	њ 	ћ 	ќ 	ѝ 	ў 	џ

U+046x 	Ѡ 	ѡ 	Ѣ 	ѣ 	Ѥ 	ѥ 	Ѧ 	ѧ 	Ѩ 	ѩ 	Ѫ 	ѫ 	Ѭ 	ѭ 	Ѯ 	ѯ

U+047x 	Ѱ 	ѱ 	Ѳ 	ѳ 	Ѵ 	ѵ 	Ѷ 	ѷ 	Ѹ 	ѹ 	Ѻ 	ѻ 	Ѽ 	ѽ 	Ѿ 	ѿ

U+048x 	Ҁ 	ҁ 	҂ 	 ҃ 	 ҄ 	 ҅ 	 ҆ 	 ҇ 	 ҈ 	 ҉ 	Ҋ 	ҋ 	Ҍ 	ҍ 	Ҏ 	ҏ

U+049x 	Ґ 	ґ 	Ғ 	ғ 	Ҕ 	ҕ 	Җ 	җ 	Ҙ 	ҙ 	Қ 	қ 	Ҝ 	ҝ 	Ҟ 	ҟ

U+04Ax 	Ҡ 	ҡ 	Ң 	ң 	Ҥ 	ҥ 	Ҧ 	ҧ 	Ҩ 	ҩ 	Ҫ 	ҫ 	Ҭ 	ҭ 	Ү 	ү

U+04Bx 	Ұ 	ұ 	Ҳ 	ҳ 	Ҵ 	ҵ 	Ҷ 	ҷ 	Ҹ 	ҹ 	Һ 	һ 	Ҽ 	ҽ 	Ҿ 	ҿ

U+04Cx 	Ӏ 	Ӂ 	ӂ 	Ӄ 	ӄ 	Ӆ 	ӆ 	Ӈ 	ӈ 	Ӊ 	ӊ 	Ӌ 	ӌ 	Ӎ 	ӎ 	ӏ

U+04Dx 	Ӑ 	ӑ 	Ӓ 	ӓ 	Ӕ 	ӕ 	Ӗ 	ӗ 	Ә 	ә 	Ӛ 	ӛ 	Ӝ 	ӝ 	Ӟ 	ӟ

U+04Ex 	Ӡ 	ӡ 	Ӣ 	ӣ 	Ӥ 	ӥ 	Ӧ 	ӧ 	Ө 	ө 	Ӫ 	ӫ 	Ӭ 	ӭ 	Ӯ 	ӯ

U+04Fx 	Ӱ 	ӱ 	Ӳ 	ӳ 	Ӵ 	ӵ 	Ӷ 	ӷ 	Ӹ 	ӹ 	Ӻ 	ӻ 	Ӽ 	ӽ 	Ӿ 	ӿ

Cyrillic Supplement

U+050x 	Ԁ 	ԁ 	Ԃ 	ԃ 	Ԅ 	ԅ 	Ԇ 	ԇ 	Ԉ 	ԉ 	Ԋ 	ԋ 	Ԍ 	ԍ 	Ԏ 	ԏ

U+051x 	Ԑ 	ԑ 	Ԓ 	ԓ 	Ԕ 	ԕ 	Ԗ 	ԗ 	Ԙ 	ԙ 	Ԛ 	ԛ 	Ԝ 	ԝ 	Ԟ 	ԟ

U+052x 	Ԡ 	ԡ 	Ԣ 	ԣ 	Ԥ 	ԥ 	Ԧ 	ԧ 	Ԩ 	ԩ 	Ԫ 	ԫ 	Ԭ 	ԭ 	Ԯ 	ԯ

Cyrillic Extended-A

U+2DEx 	 ⷠ 	 ⷡ 	 ⷢ 	 ⷣ 	 ⷤ 	 ⷥ 	 ⷦ 	 ⷧ 	 ⷨ 	 ⷩ 	 ⷪ 	 ⷫ 	 ⷬ 	 ⷭ 	 ⷮ 	 ⷯ

U+2DFx 	 ⷰ 	 ⷱ 	 ⷲ 	 ⷳ 	 ⷴ 	 ⷵ 	 ⷶ 	 ⷷ 	 ⷸ 	 ⷹ 	 ⷺ 	 ⷻ 	 ⷼ 	 ⷽ 	 ⷾ 	 ⷿ

Cyrillic Extended-B

U+A64x 	Ꙁ 	ꙁ 	Ꙃ 	ꙃ 	Ꙅ 	ꙅ 	Ꙇ 	ꙇ 	Ꙉ 	ꙉ 	Ꙋ 	ꙋ 	Ꙍ 	ꙍ 	Ꙏ 	ꙏ

U+A65x 	Ꙑ 	ꙑ 	Ꙓ 	ꙓ 	Ꙕ 	ꙕ 	Ꙗ 	ꙗ 	Ꙙ 	ꙙ 	Ꙛ 	ꙛ 	Ꙝ 	ꙝ 	Ꙟ 	ꙟ

U+A66x 	Ꙡ 	ꙡ 	Ꙣ 	ꙣ 	Ꙥ 	ꙥ 	Ꙧ 	ꙧ 	Ꙩ 	ꙩ 	Ꙫ 	ꙫ 	Ꙭ 	ꙭ 	ꙮ 	 ꙯

U+A67x 	 ꙰ 	 ꙱ 	 ꙲ 	꙳ 	 ꙴ 	 ꙵ 	 ꙶ 	 ꙷ 	 ꙸ 	 ꙹ 	 ꙺ 	 ꙻ 	 ꙼ 	 ꙽ 	꙾ 	ꙿ

U+A68x 	Ꚁ 	ꚁ 	Ꚃ 	ꚃ 	Ꚅ 	ꚅ 	Ꚇ 	ꚇ 	Ꚉ 	ꚉ 	Ꚋ 	ꚋ 	Ꚍ 	ꚍ 	Ꚏ 	ꚏ

U+A69x 	Ꚑ 	ꚑ 	Ꚓ 	ꚓ 	Ꚕ 	ꚕ 	Ꚗ 	ꚗ 	Ꚙ 	ꚙ 	Ꚛ 	ꚛ 	ꚜ 	ꚝ 	ꚞ 	ꚟ

\subsection{Georgian}
Georgian is a Unicode block containing the Mkhedruli and Asomtavruli Georgian characters used to write Modern Georgian, Svan, and Mingrelian languages. Another lower case, Nuskhuri, is encoded in a separate Georgian Supplement block, which is used with the Asomtavruli to write the ecclesiastical Khutsuri Georgian script.

U+10Ax 	Ⴀ 	Ⴁ 	Ⴂ 	Ⴃ 	Ⴄ 	Ⴅ 	Ⴆ 	Ⴇ 	Ⴈ 	Ⴉ 	Ⴊ 	Ⴋ 	Ⴌ 	Ⴍ 	Ⴎ 	Ⴏ

U+10Bx 	Ⴐ 	Ⴑ 	Ⴒ 	Ⴓ 	Ⴔ 	Ⴕ 	Ⴖ 	Ⴗ 	Ⴘ 	Ⴙ 	Ⴚ 	Ⴛ 	Ⴜ 	Ⴝ 	Ⴞ 	Ⴟ

U+10Cx 	Ⴠ 	Ⴡ 	Ⴢ 	Ⴣ 	Ⴤ 	Ⴥ

U+10Dx 	ა 	ბ 	გ 	დ 	ე 	ვ 	ზ 	თ 	ი 	კ 	ლ 	მ 	ნ 	ო 	პ 	ჟ

U+10Ex 	რ 	ს 	ტ 	უ 	ფ 	ქ 	ღ 	ყ 	შ 	ჩ 	ც 	ძ 	წ 	ჭ 	ხ 	ჯ

U+10Fx 	ჰ 	ჱ 	ჲ 	ჳ 	ჴ 	ჵ 	ჶ 	ჷ 	ჸ 	ჹ 	ჺ 	჻ 	ჼ

\subsection{Syriac and Imperial Aramaic}
\url{https://en.wikipedia.org/wiki/Aramaic_alphabet}

Syriac

U+070x 	܀ 	܁ 	܂ 	܃ 	܄ 	܅ 	܆ 	܇ 	܈ 	܉ 	܊ 	܋ 	܌ 	܍ 	

U+071x 	ܐ 	ܑ 	ܒ 	ܓ 	ܔ 	ܕ 	ܖ 	ܗ 	ܘ 	ܙ 	ܚ 	ܛ 	ܜ 	ܝ 	ܞ 	ܟ

U+072x 	ܠ 	ܡ 	ܢ 	ܣ 	ܤ 	ܥ 	ܦ 	ܧ 	ܨ 	ܩ 	ܪ 	ܫ 	ܬ 	ܭ 	ܮ 	ܯ

U+073x 	ܰ 	ܱ 	ܲ 	ܳ 	ܴ 	ܵ 	ܶ 	ܷ 	ܸ 	ܹ 	ܺ 	ܻ 	ܼ 	ܽ 	ܾ 	ܿ

U+074x 	݀ 	݁ 	݂ 	݃ 	݄ 	݅ 	݆ 	݇ 	݈ 	݉ 	݊ 			ݍ 	ݎ 	

Imperial Aramaic

U+1084x 	𐡀 	𐡁 	𐡂 	𐡃 	𐡄 	𐡅 	𐡆 	𐡇 	𐡈 	𐡉 	𐡊 	𐡋 	𐡌 	𐡍 	𐡎 	𐡏

U+1085x 	𐡐 	𐡑 	𐡒 	𐡓 	𐡔 	𐡕 		𐡗 	𐡘 	𐡙 	𐡚 	𐡛 	𐡜 	𐡝 	𐡞 	𐡟

\section{Languages}

\subsection{Íslenska}
Hver maður er borinn frjáls og jafn öðrum að virðingu og réttindum.
 
\subsection{Русский}  
Все люди рождаются свободными и равными в своем достоинстве и 
правах.
 
\subsection{Tiếng Việt}
Tất cả mọi người sinh ra đều được tự do và bình đẳng về nhân phẩm và 
quyền lợi.
 
\subsection{Ελληνικά} 
Ὅλοι οἱ ἄνθρωποι γεννιοῦνται ἐλεύθεροι καὶ ἴσοι στὴν ἀξιοπρέπεια 
καὶ τὰ δικαιώματα.

\subsection{German}
äöüßÄÖÜ Bei der UTF-8-Kodierung wird jedem Unicode-Zeichen eine speziell kodierte Zeichenkette variabler Länge zugeordnet. Dabei unterstützt UTF-8 Zeichenketten bis zu einer Länge von vier Byte, auf die sich – wie bei allen UTF-Formaten – alle Unicode-Zeichen abbilden lassen.

\subsection{Danish}
UTF-8 er en ægte udvidelse af ASCII standarden, hvilket betyder at en ASCII tekst ikke skal konverteres men også er en UTF-8 tekst.

\subsection{French}
UTF-8 (abréviation de l’anglais Universal Character Set Transformation Format - 8 bits) est un codage de caractères informatiques conçu pour coder l’ensemble des caractères du « répertoire universel de caractères codés », initialement développé par l’ISO dans la norme internationale ISO/CEI 10646, aujourd’hui totalement compatible avec le standard Unicode, en restant compatible avec la norme ASCII limitée à l’anglais de base (et quelques autres langues beaucoup moins fréquentes), mais très largement répandue depuis des décennies.

\subsection{Swedish}
UTF-8 (åtta-bitars Unicode transformationsformat) är en längdvarierande teckenkodning som används för att representera text kodad i Unicode, som en sekvens av byte (oktetter). Unicode använder upp till 21 bitar per tecken, vilket inte får plats i en byte, och därför används till exempel i textfiler vanligen en av metoderna UTF-8 eller UTF-16 för att få en serie bytes.

\subsection{Turkish}
UTF-8 8-bitlik bir Unicode dönüşüm biçimidir (İng: Unicode Transformation Format 'ın kısaltması). Unicode karakterlerini değişken sayıda 8 bitten oluşan bayt (kod birimi) gruplarıyla kodlamakta kullanılır. Rob Pike ve Ken Thompson tarafından geliştirilmiştir. UTF-8 kodlaması Unicode karakterlerini 1-6 bayt uzunluğunda diziler olarak kodlar. ASCII kodlaması içinde 0-127 arasında kalan karakterler, Unicode standardında aynı kod noktalarıyla ifade edildiğinden aynen kendi kodları ile kullanılır, diğerleri ise bayt dizileri haline gelir.

\subsection{Spanish}
Errores de codificación: Las normas de codificación establecen, por lo tanto, límites a las cadenas que se pueden formar. Según la norma, un intérprete de cadenas debe rechazar como inválidos, y no tratar de interpretar, las caracteres mal formados. Un intérprete de cadenas UTF-8 puede cancelar el proceso señalando un error, omitir los caracteres mal formados o reemplazarlos por un carácter U+FFFD (REPLACEMENT CHARACTER).

\subsection{Polish}
Przykład: Kodowanie na podstawie znaku euro €: Znak € w Unicode ma oznaczenie U+20AC. Zgodnie z informacjami w poprzednim podrozdziale taka wartość jest możliwa do zakodowania na 3 bajtach. Liczba szesnastkowa 20AC to binarnie 0010 0000 1010 1100 po uzupełnieniu wiodącymi zerami do 16 bitów, ponieważ tyle bitów trzeba zakodować na 3 bajtach w UTF-8.    Kodowanie na trzech bajtach wymaga użycia w pierwszym bajcie trzech wiodących bitów ustawionych na 1, a czwartego na 0 (1110…). Pozostałe bity pierwszego bajtu pochodzą z najstarszych czterech bitów kodowanej wartości w Unicode, co daje (1110 0010), a reszta bitów dzielona jest na dwa bloki po 6 bitów każdy (…0000 1010 1100). Do tych bloków dodawane są wiodące bity 10, by tworzyły następujące 8-bitowe wartości 1000 0010 i 1010 1100). W ten sposób rezultatem są trzy bajty w postaci 1110 0010 1000 0010 1010 1100, co w systemie szesnastkowych przyjmuje postać E2 82 AC.


\section{Misc}
\subsection{Currency Symbols}

U+20Ax 	₠ 	₡ 	₢ 	₣ 	₤ 	₥ 	₦ 	₧ 	₨ 	₩ 	₪ 	₫ 	€ 	₭ 	₮ 	₯

U+20Bx 	₰ 	₱ 	₲ 	₳ 	₴ 	₵ 	₶ 	₷ 	₸ 	₹ 	₺

\subsection{Mathematical Operators}
U+220x 	∀ 	∁ 	∂ 	∃ 	∄ 	∅ 	∆ 	∇ 	∈ 	∉ 	∊ 	∋ 	∌ 	∍ 	∎ 	∏

U+221x 	∐ 	∑ 	− 	∓ 	∔ 	∕ 	∖ 	∗ 	∘ 	∙ 	√ 	∛ 	∜ 	∝ 	∞ 	∟

U+222x 	∠ 	∡ 	∢ 	∣ 	∤ 	∥ 	∦ 	∧ 	∨ 	∩ 	∪ 	∫ 	∬ 	∭ 	∮ 	∯

U+223x 	∰ 	∱ 	∲ 	∳ 	∴ 	∵ 	∶ 	∷ 	∸ 	∹ 	∺ 	∻ 	∼ 	∽ 	∾ 	∿

U+224x 	≀ 	≁ 	≂ 	≃ 	≄ 	≅ 	≆ 	≇ 	≈ 	≉ 	≊ 	≋ 	≌ 	≍ 	≎ 	≏

U+225x 	≐ 	≑ 	≒ 	≓ 	≔ 	≕ 	≖ 	≗ 	≘ 	≙ 	≚ 	≛ 	≜ 	≝ 	≞ 	≟

U+226x 	≠ 	≡ 	≢ 	≣ 	≤ 	≥ 	≦ 	≧ 	≨ 	≩ 	≪ 	≫ 	≬ 	≭ 	≮ 	≯

U+227x 	≰ 	≱ 	≲ 	≳ 	≴ 	≵ 	≶ 	≷ 	≸ 	≹ 	≺ 	≻ 	≼ 	≽ 	≾ 	≿

U+228x 	⊀ 	⊁ 	⊂ 	⊃ 	⊄ 	⊅ 	⊆ 	⊇ 	⊈ 	⊉ 	⊊ 	⊋ 	⊌ 	⊍ 	⊎ 	⊏

U+229x 	⊐ 	⊑ 	⊒ 	⊓ 	⊔ 	⊕ 	⊖ 	⊗ 	⊘ 	⊙ 	⊚ 	⊛ 	⊜ 	⊝ 	⊞ 	⊟

U+22Ax 	⊠ 	⊡ 	⊢ 	⊣ 	⊤ 	⊥ 	⊦ 	⊧ 	⊨ 	⊩ 	⊪ 	⊫ 	⊬ 	⊭ 	⊮ 	⊯

U+22Bx 	⊰ 	⊱ 	⊲ 	⊳ 	⊴ 	⊵ 	⊶ 	⊷ 	⊸ 	⊹ 	⊺ 	⊻ 	⊼ 	⊽ 	⊾ 	⊿

U+22Cx 	⋀ 	⋁ 	⋂ 	⋃ 	⋄ 	⋅ 	⋆ 	⋇ 	⋈ 	⋉ 	⋊ 	⋋ 	⋌ 	⋍ 	⋎ 	⋏

U+22Dx 	⋐ 	⋑ 	⋒ 	⋓ 	⋔ 	⋕ 	⋖ 	⋗ 	⋘ 	⋙ 	⋚ 	⋛ 	⋜ 	⋝ 	⋞ 	⋟

U+22Ex 	⋠ 	⋡ 	⋢ 	⋣ 	⋤ 	⋥ 	⋦ 	⋧ 	⋨ 	⋩ 	⋪ 	⋫ 	⋬ 	⋭ 	⋮ 	⋯

U+22Fx 	⋰ 	⋱ 	⋲ 	⋳ 	⋴ 	⋵ 	⋶ 	⋷ 	⋸ 	⋹ 	⋺ 	⋻ 	⋼ 	⋽ 	⋾ 	⋿

\section{Ancient Scripts}
\subsection{Gothic}
Gothic is a Unicode block containing characters for writing the East Germanic Gothic language.

U+1033x 	𐌰 	𐌱 	𐌲 	𐌳 	𐌴 	𐌵 	𐌶 	𐌷 	𐌸 	𐌹 	𐌺 	𐌻 	𐌼 	𐌽 	𐌾 	𐌿

U+1034x 	𐍀 	𐍁 	𐍂 	𐍃 	𐍄 	𐍅 	𐍆 	𐍇 	𐍈 	𐍉 	𐍊
