\chapter{Heading in Booktype}

NORMAL PARAGRAPH IN BOOKTYPE this chapter is meant to show all stuff that can be done in Booktype in XeTex. Left Munich at 8:35 P. M., on 1st May, arriving at Vienna early next morning; should have arrived at 6:46.

\section{SubHeading in Booktype}

\subsection{SubSubHeading in Booktype}

Pre-formatted
\begin{verbatim}
This      is      a         table
using     white   spaces    only!
\end{verbatim}

QUOTE
\begin{quote}
QUOTE \lipsum[1]
\end{quote}
\begin{quotation}
QUOTATION arrived at 6:46, but train was an hour late. Buda-Pesth seems a wonderful place, from the glimpse which I got of it from the train and the 

little I could walk through the streets. I feared to go very far from the station, as 

we had arrived late and would start as near the correct time as possible. The impression I had was that we were leaving the West and ent.
\end{quotation}
\begin{verse}
VERSE arrived at 6:46, but train was an hour late. Buda-Pesth seems a wonderful place, from the glimpse which I got of it from the train and 

the little I could walk through the streets. 

I feared to go very far from the station, as we had arrived late and would start as near the correct time as possible. 

The impression I had was that we were leaving the West and ent
\end{verse}

\subsubsection{Font formatting}
Text you want \underline{underlined goes here.} \blindtext


I want to \emph{emphasize} a word.
\emph{In this emphasized sentence, there is an emphasized \emph{word} which looks upright.}

\textbf{Bold text} may be used to heavily emphasize very important words or phrases.
\textit{Italicised text} may be used to lightly emphasize words or phrases.

You can also combing \textit{italicised and \textbf{bold text}} as shown here.
Or the other way around, combine \textbf{bold and \textit{italicised text}} as shown here.

There is \textbf{bold text and} \textmd{medium weight text} - a subtle distinction.

Monospace font like Courier is called \texttt{teletype font}.

\textsf{Sans serif font family} is being used at the beginning of this sentence.

A neat little feature is the \textsc{Small Caps Feature applied here} or the
{\scshape Small Caps Feature applied here}. 

Old style numbers look like this: {\addfontfeature{Numbers=OldStyle}1234567890}. If old style is set in the preamble then these should also be old style: 1234567890.

Of course you can also \uppercase{shout it all in uppercase}.

\tiny{tiny sample text}
\scriptsize{scriptsize sample text}
\footnotesize{footnotesize sample text}
\small{small sample text}
\normalsize{normalsize sample text}
\large{large sample text}
\Large{Large sample text}
\LARGE{LARGE sample text}
\huge{huge sample text}
\Huge{Huge sample text}

\normalsize

\subsubsection{Links, URLs}
Here's a link to \href{http://twitter.com/home}{Twitter}. 
And here is a link to \url{http://booktype.pro}

\subsubsection{Left align}
\begin{flushleft}
\lipsum[1]
\end{flushleft}

\subsubsection{Right align}
\begin{flushright}
\lipsum[1]
\end{flushright}

\subsubsection{Center align}
\begin{center}
\lipsum[1]
\end{center}

\subsubsection{Justify / Blocksatz}
Back to normal, which is justified text. 
\lipsum[1]

\subsubsection{indent a paragraph}

\lipsum[1]

\begin{adjustwidth}{2cm}{}
\lipsum[1]
\end{adjustwidth}

\begin{adjustwidth}{4cm}{}
\lipsum[1]
\end{adjustwidth}

\begin{adjustwidth}{2cm}{}
\lipsum[1]
\end{adjustwidth}

\lipsum[1]

\subsubsection{Lists}

\begin{itemize}  
\item The first item 
\item The second item 
\item The third etc \ldots 
\end{itemize}

\begin{enumerate}  
\item The first item 
\item The second item 
\item The third etc \ldots 
\end{enumerate}

\begin{itemize}
\item \blindtext
\item \blindtext
\end{itemize}

\begin{enumerate}
\item \blindtext
\item \blindtext
\end{enumerate}

\subsubsection{Font sizes}
\font\FontSizeTenPT="Ubuntu" at 10pt
\font\testhuge="Ubuntu" at 100pt
\font\testbig="Ubuntu" at 80pt
\font\testnormal="Ubuntu" at 50pt
\font\testsmall="Ubuntu" at 40pt
\font\testtiny="Ubuntu" at 24pt

\offinterlineskip\testnormal

{\FontSizeTenPT 10pt is the size of this}
{\testhuge ABCDE}
\vskip1pt
{\testbig ABCDE}
\vskip1pt
ABCDE
\vskip1pt
{\testsmall ABCDE}
\vskip1pt
{\testtiny ABCDE}

\normalsize 
\chapter{Chapter title and text examples}
Paragraph style "Normal Text" Lorem ipsum dolor sit amet,
consetetur sadipscing elitr, sed diam nonumy eirmod tempor invidunt
ut labore et dolore magna aliquyam erat, sed diam voluptua. At vero
eos et accusam et justo duo dolores et ea rebum. Stet clita kasd
gubergren, no sea takimata sanctus est Lorem ipsum dolor sit amet.
Lorem ipsum dolor sit amet, consetetur sadipscing elitr, sed diam
nonumy eirmod tempor invidunt ut labore et dolore magna aliquyam
erat, sed diam voluptua. At vero eos et accusam et justo duo
dolores et ea rebum. Stet clita kasd gubergren, no sea takimata
sanctus est Lorem ipsum dolor sit amet.

\begin{quote}
Paragraph style "Quote" Lorem ipsum dolor sit
amet, consetetur sadipscing elitr, sed diam nonumy (press
Shift+Return for single line breaks)\\
eirmod tempor invidunt
\end{quote}

Now we will do some font size. Booktype gives me the
option to select one of the following font sizes:

\font\FontSizeTenPT="Ubuntu" at 10pt
\font\FontSizeTwelvePT="Ubuntu" at 12pt
\font\FontSizeFourteenPT="Ubuntu" at 14pt
\font\FontSizeSixteenPT="Ubuntu" at 16pt
\font\FontSizeEighteenPT="Ubuntu" at 18pt
\font\FontSizeTwentyPT="Ubuntu" at 20pt
\font\FontSizeTwentytwoPT="Ubuntu" at 22pt
\font\FontSizeTwentyfourPT="Ubuntu" at 24pt
\font\FontSizeTwentysixPT="Ubuntu" at 26pt
\font\FontSizeTwentyeightPT="Ubuntu" at 28pt
\font\FontSizeThirtyPT="Ubuntu" at 30pt

{\FontSizeTenPT 10pt is the size of this paragraph.}
{\FontSizeTwelvePT 12pt is the size of this paragraph.}
{\FontSizeFourteenPT 14pt is the size of this paragraph.}
{\FontSizeSixteenPT 16pt is the size of this paragraph.}
{\FontSizeEighteenPT 18pt is the size of this paragraph.}
{\FontSizeTwentyPT 20pt is the size of this paragraph.}
{\FontSizeTwentytwoPT 22pt is the size of this paragraph.}
{\FontSizeTwentyfourPT 24pt is the size of this paragraph.}
{\FontSizeTwentysixPT 26pt is the size of this paragraph.}
{\FontSizeTwentyeightPT 28pt is the size of this paragraph.}
{\FontSizeThirtyPT 30pt is the size of this paragraph.}

\chapter{Style "Heading" and colours}

Paragraph style "Normal Text" with a couple of font
styles. Here is 
\textbf{some bold text}. And here is 
\textit{some italicised text}. 
Part of this sentence is underlined. 
We can also combine these, starting 
\textit{italicise now, \textbf{then some bold}, back to italics only} and back to normal. This can
also happen the other way around: 
\textbf{starting bold now, \textit{then some italics}, back to bold only} and
back to normal.

% http://www.andy-roberts.net/writing/latex/tables
\noindent
\begin{table}
	% hide table numbering if set for list of tables or chapters in general
	% set in theme variables
	\ifnum\pdfstrcmp{\varShowToTablesNumbering}{false}=0 
		\caption*{Just a few names in this table}
	\else
		\ifnum\pdfstrcmp{\varChapterWriteNumbering}{false}=0 
			\caption*{Just a few names in this table}
		\else
			\caption{Just a few names in this table} 
		\fi
	\fi
	\begin{center}
		\begin{tabular}{lllll}
			\toprule
			Feature & Icon & Description & HTML & LaTeX \\
			\midrule
			Bold & Fat B & Makes the font look fatter & <b> or <strong> & textbf \\
			Bold & Fat B & Makes the font look fatter & <b> or <strong> & textbf \\
			\bottomrule
		\end{tabular}
	\end{center}
\end{table}

Here is a coloured piece of text: 
\textcolor[rgb]{1,0,0}{This is \textbf{red}.}
\textcolor[rgb]{0,0,1}{And this is \textbf{blue}.}
\textcolor[rgb]{0,1,0}{This is \textit{green}.}
\textcolor[rgb]{0.5,0.5,0.5}{\textbf{And pale are you}.}

\section{Style "Subheading", tables and links}
At vero eos et accusam et justo duo dolores et ea rebum. Stet
clita kasd gubergren, no sea takimata sanctus est Lorem ipsum dolor
sit amet. Lorem ipsum dolor sit amet, consetetur sadipscing elitr.
And let's now insert a table:
